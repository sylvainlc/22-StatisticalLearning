\documentclass[a4paper,10pt,fleqn]{article}

\usepackage{a4wide,amsmath,amsthm,amssymb,bbm,fancyhdr}
\usepackage{ifthen,color,enumerate,comment,dsfont,pdfsync,framed,todonotes,enumitem}
\newboolean{corrige}
\setboolean{corrige}{true}

\newcommand{\titre}[1]{\textbf{\textsc{#1}}}

\RequirePackage[T1]{fontenc}

\usepackage[latin1]{inputenc}
\usepackage{graphicx}
\usepackage{dsfont}
\newcommand{\thisyear}{}
\usepackage{enumitem}
\newcommand{\eqsp}{\,}
\newcommand{\R}{\ensuremath{\mathbb{R}}}
\newcommand{\calF}{\mathcal{F}}
\newcommand{\N}{\mathbb{N}}
\newcommand{\rset}{\ensuremath{\mathbb{R}}}
\renewcommand{\P}{\ensuremath{\operatorname{P}}}
\newcommand{\bP}{\mathbb{P}}
\newcommand{\E}{\ensuremath{\mathbb{E}}}
\newcommand{\rme}{\ensuremath{\mathrm{e}}}
\newcommand{\calH}{\ensuremath{\mathcal{H}}}
\newcommand{\xset}{\ensuremath{\mathsf{X}}}
\newcommand{\V}{\ensuremath{\mathbb{V}}}
\newcommand{\Sb}{\ensuremath{\mathbb{S}}}
\newcommand{\gaus}{\ensuremath{\mathcal{N}}}
\newcommand{\HH}{\ensuremath{\mathcal{H}}}
\newcommand{\F}{\ensuremath{\mathcal{F}}}
\newcommand{\W}{\ensuremath{\mathcal{W}}}
\newcommand{\X}{\ensuremath{\mathcal{X}}}
\newcommand{\1}{\ensuremath{\mathbbm{1}}}
\newcommand{\dlim}{\ensuremath{\stackrel{\mathcal{L}}{\longrightarrow}}}
\newcommand{\plim}{\ensuremath{\stackrel{\mathrm{P}}{\longrightarrow}}}
\newcommand{\PP}{\ensuremath{\mathbb{P}}}
\newcommand{\p}{\ensuremath{\mathbb{P}}}
\newcommand{\eps}{\varepsilon}
\newcommand{\bE}{\mathbb{E}}
\newcommand{\pa}[1]{\left(#1\right)}
\newcommand{\hatk}{\widehat K}
\newcommand{\f}{\varphi}
\newcommand{\Id}{\textsf{Id}}
\newcommand{\bfU}{\mathbf{U}}
\newcommand{\bfX}{\mathbf{X}}
\newcommand{\bfs}{\mathbf{\Sigma}}
\newcommand{\bfA}{\mathbf{A}}
\newcommand{\bfV}{\mathbf{V}}
\newcommand{\bfB}{\mathbf{B}}
\newcommand{\bfI}{\mathbf{I}}
\newcommand{\bfD}{\mathbf{D}}
\newcommand{\bfK}{\mathbf{K}}
\newcommand{\argmin}{\mathop{\textrm{argmin}}}
\newcommand{\argmax}{\mathop{\textrm{argmax}}}
\newcommand{\crit}{\mathop{\textrm{crit}}}
\newcommand{\C}{\mathcal{C}}
\newcommand{\pc}{\pi_{\mathcal{C}}}


% Style
%\pagestyle{fancyplain}
\renewcommand{\sectionmark}[1]{\markright{#1}}
\renewcommand{\subsectionmark}[1]{}
%\lhead[\fancyplain{}{\thepage}]{\fancyplain{}{\footnotesize {\sf
%MAT4506 Introduction to Machine Learning  %/ \rightmark
%}}}
%\rhead[\fancyplain{}{\footnotesize {\sf MAT4506 Introduction to machine learning, \thisyear %/ \rightmark
%}}]{\fancyplain{}{\thepage}}


\newtheorem{theorem}{Theorem}

%% Titre
%\title{{\bf Machine learning}}
%\author{{\em Logistic regression}}
%\date{}


\begin{document}

%\noindent Machine learning \hfill ISUP - Sorbonne Universit\'e \\
% 2022-2023

\noindent\hrulefill

\begin{center}
\textsc{kernels}
\end{center}
\hrulefill

\medskip



\section{Warm-up}
 Let $\calH$ be a RKHS associated with a positive definite kernel $k: \xset\times \xset \to \rset$.
\begin{enumerate}
\item  Prove that for all $(x,y)\in\xset\times \xset$ and $f \in \calH$, 
$$
|f(x)-f(y)|\leqslant \|f\|_{\calH}\|k(x,\cdot)-k(y,\cdot)\|_{\calH}\eqsp.
$$

%\vspace{.2cm}
%
%{\em
%The proof follows from Cauchy-Schwarz inequality as, for all $(x,y)\in\xset^2$,
%$$
%|f(x)-f(y)|= |\langle f, k(x,\cdot)\rangle_{\mathcal{H}}-\langle f, k(x,\cdot)\rangle_{\mathcal{H}}| = |\langle f, k(x,\cdot)-k(y,\cdot)\rangle_{\mathcal{H}}|\eqsp.
%$$
%}

\item  Prove that the kernel $k$ associated with $\calH$ is unique, i.e.\ if $\widetilde k$ is another potitive definite kernel satisfying the RKHS properties for $\calH$, then $k = \widetilde k$.

%\vspace{.2cm}
%
%{\em
%Write, for all $x\in\xset$,
%$$
%\|k(x,\cdot) - \widetilde k(x,\cdot)\|_{\calH}^2 = \langle k(x,\cdot) - \widetilde k(x,\cdot),k(x,\cdot) - \widetilde k(x,\cdot)\rangle = k(x,x) - \widetilde k(x,x) +  \widetilde k(x,x) - k(x,x)= 0\,.
%$$
%}

\item  Prove that  for all $x\in\xset$, the function defined on $\calH$ by $\delta_x: f \mapsto f(x)$ is continuous.

\vspace{.2cm}

{\em

}
\end{enumerate}

\section{Kernel Ridge regression}
Let $\mathcal{H}$ be a RKHS on $\mathcal{X}$ with kernel $k$.  We consider the regression model $Y_{i}=f^*(X_{i})+\xi_{i}$, $i\in\{1,\ldots,n\}$, with $\xi_{i}$, $1\leq i \leq n$,  independent centered noise with finite variance. The unknown function $f^*$ is estimated  by the solution 
$\widehat f$  of the convex minimization problem
$$\widehat f=\argmin_{f\in\mathcal{H}}\left\{{1\over n}\sum_{i=1}^n(Y_{i}-f(X_{i}))^2+{\lambda\over n} \|f\|_{\mathcal{H}}^2\right\}\,,$$
with $\lambda>0$.

\subsection{Solving Kernel ridge regression}
\begin{enumerate}
	%\item Prove that $\widehat f$ belongs to the linear span $V=\textsf{span}\{k(x_{i},.): i=1,\ldots,n\}$.
	\item Check that $\widehat f:x\mapsto \sum_{j=1}^n\widehat \beta_{j}k(X_{j},x)$ where $\widehat \beta=(\widehat \beta_{1},\ldots,\widehat \beta_{n})^\top$ is solution to
	$$ \widehat \beta=\argmin_{\beta\in\R^n}\left\{\|Y-K\beta\|^2+{\lambda} \beta^\top K\beta\right\}$$
	with $K$ defined by $K=(k(X_{i},X_{j}))_{1\leq i,j\leq n}$. Comment on this result.

%\vspace{.2cm}
%
%{\em
%There exists $ \beta$ such that, for all $x$,
%$$
%			\widehat{f}(x) = \sum_{j=1}^n \beta_j k(X_j, x).
%$$
%This yields
%		\begin{align*}
%			{1\over n}\sum_{i=1}^n(Y_{i}-f(X_{i}))^2+{\lambda\over n} \|f\|_{\mathcal{H}}^2 
%			& = {1\over n}\sum_{i=1}^n(Y_{i}-\sum_{j=1}^n \beta_j k(X_j, X_i))^2+{\lambda\over n} \langle \sum_{j=1}^n \beta_j k(X_j, \cdot), \sum_{i=1}^n \beta_i k(X_i, \cdot) \rangle,
%		\end{align*}
%	which gives the result, since 
%$$
%		\langle \sum_{j=1}^n \beta_j k(X_j, \cdot), \sum_{i=1}^n \beta_i k(X_i, \cdot) \rangle  = \sum_{i,j=1}^n \beta_i \beta_j k(X_i, X_j) = \beta^\top K \beta.
%$$
%}
	\item Assume that $K$ is non-singular. Give an explicit expression for $\widehat \beta$.

%\vspace{.2cm}
%
%{\em
%Write, for all $\beta$,
%$$
%			L(\beta) = \| Y - K \beta\|_2^2 + \lambda \beta^\top K \beta.
%$$
%		The gradient of $L$ is then given by
%		\begin{align*}
%			\nabla L (\beta) & = -2K^\top (Y - K \beta) + \lambda (K \beta + K^\top \beta)  \\
%			& = -2K(Y-K \beta) + 2 \lambda K \beta.
%	\end{align*}
%	The minimum $\widehat{\beta}$ of $L$ satisfies 
%	\begin{align*}
%		\Leftrightarrow & -2K(Y-K \widehat{\beta}) + 2 \lambda K \widehat{\beta} = 0\\
%		\Leftrightarrow & \widehat{\beta} = (K + \lambda I)^{-1} Y.
%	\end{align*}
%}
\end{enumerate}

\subsection{Bias and variance}
%The design $x_{1},\ldots,x_{n}$ is assumed to be fixed.
We assume that $f^*\in\mathcal{H}$ and we write 
$$f^*_{V}: x \mapsto \sum_{i=1}^n \beta_{i}^*k(X_{i},x)$$ 
for the projection of $f^*$ onto the linear span $V=\textrm{span}\{k(X_{i},.): i=1,\ldots,n\}$, with respect to the Hilbert norm $\|\cdot\|_{\mathcal{H}}$. We write $K=\sum_{i=1}^n \lambda_{i}u_{i}u_{i}^\top$ for an eigenvalue decomposition of $K$.
\begin{enumerate}
	\item Check that 
	$$K\widehat \beta= \sum_{i=1}^n {\lambda_{i}\over \lambda_{i}+\lambda} \langle Y, u_{i}\rangle u_{i}\quad\textrm{with}\quad Y=(Y_{1},\ldots,Y_{n})^\top\,.$$

%\vspace{.2cm}
%
%{\em
%Since $(u_i)_{1 \leq i \leq n}$ is an orthonormal basis of $\R^n$,   
%		\begin{align*}
%			K \widehat{\beta} & = \sum_{i=1}^n \langle K \widehat{\beta}, u_i\rangle u_i\\
%			& = \sum_{i=1}^n \langle K (K+\lambda I)^{-1}Y, u_i\rangle u_i\\
%			& = \sum_{i=1}^n \langle Y, (K+\lambda I)^{-1} K  u_i\rangle u_i\\
%			& = \sum_{i=1}^n \frac{\lambda_i }{\lambda + \lambda_i} \langle Y, u_i \rangle u_i.
%		\end{align*}
%}
	\item Check that
	$$\|\E[K\widehat \beta]-K\beta^*\|_2^2=\sum_{i=1}^n \left( \lambda\lambda_{i}\over \lambda_{i}+\lambda\right)^2 \langle \beta^*, u_{i}\rangle^2.$$
%
%\vspace{.2cm}
%
%{\em
%First, note that, for all $1 \leq i \leq n$, 
%		\begin{align*}
%			\langle \E [Y], u_i \rangle  = \langle K \beta^* , u_i\rangle 
%			= \langle \beta^* , K u_i\rangle = \lambda_i \langle \beta^* , u_i\rangle.
%		\end{align*}
%		Consequently, 
%		\begin{align*}
%			\| \E [K \widehat{\beta}] - K \beta^* \|^2 
%			& = \left\| \sum_{i=1}^n \frac{\lambda_i}{\lambda_i + \lambda} \langle \E [Y], u_i \rangle u_i - \sum_{i=1}^n \langle K \beta^*, u_i \rangle u_i  \right\|_2^2 \\
%			& = \left\| \sum_{i=1}^n \left( \frac{\lambda_i^2}{\lambda_i + \lambda} - \lambda_i \right)  \langle \beta^*, u_i \rangle u_i \right\|_2^2 \\
%			& = \sum_{i=1}^n \left( \frac{\lambda \lambda_i }{\lambda_i + \lambda}\right)^2 \langle \beta^*, u_i \rangle^2.
%		\end{align*}
%}
	\item We assume henceforth that the $\xi_{i}=Y_{i}-f^*(X_{i})$, $i=1,\ldots,n$, have a covariance $\mathbb{V}[\xi]=\sigma^2I_{n}$.
	Check that the covariance matrix of $K\widehat \beta$ is equal to
	$$\mathbb{V}[K\widehat \beta]=\sum_{i=1}^n \left(\lambda_{i}\sigma\over \lambda_{i}+\lambda\right)^2u_{i}u_{i}^\top\,.$$

%\vspace{.2cm}
%
%{\em
%Since $\widehat{\beta} = (K + \lambda I)^{-1}y$, 
%		\begin{align*}
%			\mathbb{V}[K \widehat{\beta}] & = K \mathbb{V}[ (K + \lambda I)^{-1} Y ] K^\top\\
%			& = K (K + \lambda I)^{-1} \mathbb{V}[Y](K + \lambda I)^{-1} K\\
%			& = \sigma^2 K^2 (K + \lambda I)^{-2}\\
%			& = \sum_{i=1}^n \left( \frac{\lambda_i \sigma }{\lambda_i + \lambda}\right)^2 u_i u_i^\top,
%		\end{align*}
%	using the eigenvector decomposition of $K$.
%}
	\item We define $\|f\|_{n}^2:= {1\over n}\sum_{i=1}^nf(X_{i})^2$. Prove that
	$$\E\left[\|\widehat f - f^*\|^2_{n} \right]={1\over n}\sum_{i=1}^n \left( \lambda_{i}\over \lambda+\lambda_{i}\right)^2 \left(\lambda^2\langle \beta^*, u_{i}\rangle^2+\sigma^2\right).$$

\vspace{.2cm}

{\em

}
\end{enumerate}


\end{document} 