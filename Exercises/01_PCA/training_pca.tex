\documentclass[a4paper,10pt,fleqn]{article}

\usepackage{a4wide,amsmath,amsthm,amssymb,bbm,fancyhdr}
\usepackage{ifthen,color,enumerate,comment,dsfont,pdfsync,framed,todonotes,enumitem}

\newcommand{\titre}[1]{\textbf{\textsc{#1}}}

\RequirePackage[T1]{fontenc}

\usepackage[latin1]{inputenc}
\usepackage{graphicx}
\usepackage{dsfont}
\usepackage{enumitem}
\newcommand{\eqsp}{\,}
\newcommand{\R}{\ensuremath{\mathbb{R}}}
\newcommand{\calF}{\mathcal{F}}
\newcommand{\rmd}{\mathrm{d}}
\newcommand{\N}{\mathbb{N}}
\newcommand{\rset}{\ensuremath{\mathbb{R}}}
\renewcommand{\P}{\ensuremath{\operatorname{P}}}
\newcommand{\bP}{\mathbb{P}}
\newcommand{\E}{\ensuremath{\mathbb{E}}}
\newcommand{\rme}{\ensuremath{\mathrm{e}}}
\newcommand{\calH}{\ensuremath{\mathcal{H}}}
\newcommand{\xset}{\ensuremath{\mathsf{X}}}
\newcommand{\V}{\ensuremath{\mathbb{V}}}
\newcommand{\Sb}{\ensuremath{\mathbb{S}}}
\newcommand{\gaus}{\ensuremath{\mathcal{N}}}
\newcommand{\HH}{\ensuremath{\mathcal{H}}}
\newcommand{\F}{\ensuremath{\mathcal{F}}}
\newcommand{\W}{\ensuremath{\mathcal{W}}}
\newcommand{\X}{\ensuremath{\mathcal{X}}}
\newcommand{\1}{\ensuremath{\mathbbm{1}}}
\newcommand{\dlim}{\ensuremath{\stackrel{\mathcal{L}}{\longrightarrow}}}
\newcommand{\plim}{\ensuremath{\stackrel{\mathrm{P}}{\longrightarrow}}}
\newcommand{\PP}{\ensuremath{\mathbb{P}}}
\newcommand{\p}{\ensuremath{\mathbb{P}}}
\newcommand{\eps}{\varepsilon}
\newcommand{\bE}{\mathbb{E}}

\newcommand{\pa}[1]{\left(#1\right)}
\newcommand{\hatk}{\widehat K}
\newcommand{\f}{\varphi}
\newcommand{\Id}{\textsf{Id}}
\newcommand{\bfU}{U}
\newcommand{\bfX}{X}
\newcommand{\bfs}{\Sigma}
\newcommand{\bfA}{A}
\newcommand{\bfV}{V}
\newcommand{\bfB}{B}
\newcommand{\bfI}{\mathrm{I}}
\newcommand{\bfD}{D}
\newcommand{\bfK}{K}
\newcommand{\argmin}{\mathop{\textrm{argmin}}}
\newcommand{\argmax}{\mathop{\textrm{argmax}}}
\newcommand{\crit}{\mathop{\textrm{crit}}}
\newcommand{\C}{\mathcal{C}}
\newcommand{\pc}{\pi_{\mathcal{C}}}


% Style


\begin{document}

%\noindent EMINES \hfill \\%ISUP - Sorbonne Universit\'e \\
% 2023-2023

\noindent\hrulefill

\begin{center}
\textsc{Principal Component Analysis}
\end{center}
\hrulefill

\medskip


\section*{Warm-up}

\begin{enumerate}
\item Let $\bfA$ be a $n\times d$ matrix with real entries. Show that $\mathrm{Im}(\bfA)=\mathrm{Im}(\bfA \bfA^\top)$.
 \item Let $\{U_k\}_{1\leq k \leq r}$ be a family of $r$ orthonormal vectors of $\rset^n$.
Show that $\sum_{k=1}^{r} U_k U_k^\top$ is the matrix associated with the orthogonal projection onto $H=\{\sum_{k=1}^r \alpha_k U_k\,; \ \alpha_1,\ldots,\alpha_r \in \rset\}$. Deduce that if $\bfA$ is a $n\times d$ matrix with real entries such that each column of $\bfA$ is in $H$, then,
$$
\left(\sum_{k=1}^r U_k U_k^\top\right) \bfA=\bfA\eqsp.
$$
\end{enumerate}

\section*{Kernel Principal Component Analysis}
\subsection*{Principal Component Analysis}
 Let $(X_i)_{1\leqslant i\leqslant n}$ be i.i.d. random variables in $\rset^d$ and consider the matrix $\bfX\in\rset^{n\times d}$ such that the $i$-th row of $\bfX$ is the observation $X^\top_i$. In this exercise, it is assumed that data are preprocessed so that the columns of $\bfX$ are centered.  This means that for all $1\leqslant k \leqslant d$, $\sum_{i=1}^{n}X_{i,k} = 0$. Let $\bfs_n$ be the empirical covariance matrix:
$$
\bfs_n = n^{-1}\sum_{i=1}^n X_i X^\top_i\eqsp.
$$
Principal Component Analysis  aims at reducing the dimensionality of the observations $(X_i)_{1\leqslant i \leqslant n}$ using a {\em compression} matrix $\bfU\in \rset^{d\times p}$ with orthonormal columns with $p\leqslant d$ so that for each $1\leqslant i \leqslant n$, $\bfU^\top X_i$ ia a low dimensional representation of $X_i$. The original observation may then be partially recovered using  $\bfU\in \rset^{d\times p}$. Principal Component Analysis computes $\bfU$ using the least squares approach:
$$
\bfU_{\star} \in \underset{U\in  \rset^{d\times p}}{\mathrm{argmin}} \;\sum_{i=1}^n\|X_i - \bfU\bfU^\top X_i\|_2^2\eqsp,
$$
\begin{enumerate}
\item Prove that for all $\rset^{n \times d}$ matrix $\bfA$ with rank $r$, there exist $\sigma_1\geqslant \ldots \geqslant \sigma_r>0$ such that
$$
\bfA = \sum_{k=1}^r \sigma_k u_k v^\top_k\eqsp,
$$
where $\{u_1,\ldots,u_r\}\subset \rset^n$ and $\{v_1,\ldots,v_r \} \subset \rset^d$ are two families of orthonormal  vectors. The vectors $\{u_1,\ldots,u_r\}$  (resp. $\{v_1,\ldots,v_r\}$) are the left-singular (resp. right-singular) vectors associated with $\{\sigma_1,\ldots,\sigma_r\}$, the singular values of $\bfA$.
%{\em 
%If $\bfU$ denotes the $\rset^{n\times r}$ matrix with columns given by $\{u_1,\ldots,u_r\}$ and $\bfV$ denotes the $\rset^{d \times r}$ matrix with columns given by $\{v_1,\ldots,v_r\}$, then the singular value decomposition of $\bfA$ may also be written as
%$$
%\bfA = \bfU\bfD_r\bfV^T\eqsp,
%$$
%where $\bfD_r = \mathrm{diag}(\sigma_1,\ldots,\sigma_r)$. Then, $\bfA^T\bfA$ and $\bfA\bfA^T$ are positive semidefinite such that
%$$
%\bfA^T\bfA = \bfV\bfD_r^2\bfV^T\quad\mathrm{and}\quad \bfA\bfA^T = \bfU\bfD_r^2\bfU^T\eqsp.
%$$
%In the framwework of this exercise, $n\bfs_n = \bfX^T\bfX$ so that diagonalizing $n\bfs_n$ is equivalent to computing the singular value decomposition of $\bfX$.
%}
\item Prove that solving the PCA least squares optimization problem boils down to computing
$$
\bfU_{\star} \in \hspace{-0.5cm}\underset{\bfU\in \rset^{d\times p}\eqsp,\eqsp \bfU^\top\bfU = \bfI_p}{\mathrm{argmax}} \hspace{-.4cm}\{ \mathrm{trace}(\bfU^\top\bfs_n\bfU)\}\eqsp.
$$
\item Let $\{\vartheta_1,\ldots,\vartheta_d\}$ be orthonormal eigenvectors associated with the eigenvalues $\lambda_1\geqslant \ldots \geqslant \lambda_d$ of $\bfs_n$. Prove that a solution to this problem is given by the matrix $\bfU_{\star}$ with columns $\{\vartheta_1,\ldots,\vartheta_p\}$.
%there exists $p+1\leqslant j_0\leqslant d$ such that $\alpha_{k_0}>0$. Then,
%\begin{enumerate}
%\item either $\alpha_{j_0} + \alpha_{k_0}<1$ and then, as $\lambda_{j_0}\geqslant \lambda_{k_0}$,
%$$
%f_d:(\alpha_1,\ldots,\alpha_d) \leqslant \sum_{i=1, i\neq j_0, i\neq k_0}^{d}\alpha_i\lambda_i + (\alpha_{j_0} + \alpha_{k_0})\lambda_{j_0}\eqsp,
%$$
%\item either $\alpha_{j_0} + \alpha_{k_0}>1$ and then, as $\lambda_{j_0}\geqslant \lambda_{k_0}$,
%$$
%f_d:(\alpha_1,\ldots,\alpha_d) \leqslant \sum_{i=1, i\neq j_0, i\neq k_0}^{d}\alpha_i\lambda_i + \lambda_{j_0} + (\alpha_{j_0} + \alpha_{k_0}-1)\lambda_{k_0}\eqsp,
%$$
%\end{enumerate}
\item  For any dimension $1\leqslant p \leqslant  d$, let $\calF_d^p$ be the set of all vector subpaces of $\rset^d$ with dimension $p$. Consider the linear span $V_d$ defined as
$$
V_p \in \underset{V\in \calF_d^p}{\mathrm{argmin}} \;\sum_{i=1}^n\|X_i - \pi_V(X_i)\|_2^2\eqsp,
$$
where $\pi_V$ is the orthogonal projection onto the linear span $V$. Prove that $V_1 = \mathrm{span}\{v_1\}$ where
$$
v_1 \in \underset{v \in \rset^d\,;\, \|v\|_2=1}{\mathrm{argmax}} \sum_{i=1}^n   \langle X_i, v \rangle^2\eqsp.
$$
\item For all $2\leqslant p \leqslant d$, following the same steps, prove that a solution to the optimization problem is given by $V_p = \mathrm{span}\{v_1, \ldots, v_p\}$ where
\begin{equation}
\label{eq:vecpca}
v_1 \in \underset{v\in \rset^d\,;\,\|v\|=1}{\mathrm{argmax}} \sum_{i=1}^n\langle X_i,v\rangle^2 \quad\mbox{and for all}\;\; 2\leqslant k \leqslant p\;,\;\; v_k \in \underset{\substack{v\in \rset^d\,;\,\|v\|=1\,;\\ v\perp v_1,\ldots,v\perp v_{k-1}}}{\mathrm{argmax}}\sum_{i=1}^n\langle X_i,v\rangle^2\eqsp.
\end{equation}
\item Prove that the vectors $\{v_1, \ldots , v_k\}$ defined by \eqref{eq:vecpca} can be chosen as the orthonormal eigenvectors associated with the $k$ largest eigenvalues of the empirical covariance matrix $\bfs_n$.
\item The orthonormal eigenvectors associated with the eigenvalues of $\Sigma_n$ allow to define the principal components as follows. Then, as $V_d = \mathrm{span}\{\vartheta_1, \ldots, \vartheta_d\}$, for all $1\leqslant i\leqslant n$,
$$
\pi_{V_d}(X_i) = \sum_{k=1}^d \langle X_i,\vartheta_k\rangle \vartheta_k  = \sum_{k=1}^d (X^\top_i \vartheta_k)\vartheta_k = \sum_{k=1}^d c_k(i)\vartheta_k\eqsp,
$$
where for all $1\leqslant k \leqslant d$, the $k$-th principal component is defined as $c_k = \mathbf{X}\vartheta_k$. Prove that $(c_1,\ldots,c_d)$ are orthogonal vectors.
%\item Let $W_d$ be the vector subspace of $\rset^n$ generated by $\{c_1,\ldots,c_d\}$. Prove that
%$$
%\pi_{W_d}(X_i) = \sum_{j=1}^d \vartheta_{j}(i)c_{j}\eqsp.
%$$
%\begin{comment}
%Since $(c_j)_{1 \leqslant j\leqslant d}$ form an orthogonal basis of $W_d$, for all $1\leqslant i\leqslant n$,
%$$
%\pi_{W_d}(X_i) = \sum_{j=1}^d \frac{\langle c_{j}, X_i\rangle }{\|c_{j}\|^2 }c_{j}\eqsp.
%$$
%For all $1\leqslant j\leqslant d$, $\|c_{j}\|^2 = n \lambda_{j}$ and
%$$
%\langle c_{j}, X_i \rangle  = \langle \bfX \vartheta_{j} , X_i \rangle  = \bfX^T_i\bfX\vartheta_{j} = (\bfX^T\bfX\vartheta_{j})_i = (n \bfs_n \vartheta_{j})_i = n \lambda_{j} \vartheta_{j}(i)\eqsp.
%$$
%This yields, for all $1\leqslant i\leqslant n$,
%$$
%\pi_{W_d}(X_i) = \sum_{j=1}^d \vartheta_{j}(i)c_{j}\eqsp.
%$$
%\end{comment}
\end{enumerate}


\subsection*{Application to RKHS}
Let $(X_i)_{1\le i \le n}$ be $n$ observations in a general space $\X$ and $k: \X\times \X \to \rset$ a positive function. We assume that $k$ is symmetric and that for all $n\geqslant 1$, $(a_i)_{1\leqslant i \leqslant n}\in\rset^n$ and $(x_i)_{1\leqslant i \leqslant n}\in\X^n$, $\sum_{1\leqslant i,j \leqslant n}a_ia_j k(x_i,x_j)\geqslant 0$.  For all $x\in\X$, $\phi(x)$ denotes the function $\phi(x): y\to k(x,y)$.

Let $\W$ be a Hilbert space of real-valued functions defined on $\X$, endowed with an inner product denoted by $\langle \cdot,\cdot \rangle_{\W}$, and such that for all $x\in\X$, $\phi(x)\in\W$ and for all $f\in \W$ and all $x\in\X$, $f(x) = \langle f,\phi(x) \rangle_{\W}$. The aim is now to perform a PCA on $(\phi(X_1),\ldots,\phi(X_n))$. It is assumed that $\sum_{i=1}^n \phi(X_i) = 0$. Define
$$
\bfK = \left(k(X_i,X_j)\right)_{1\leqslant i,j \leqslant n}\eqsp.
$$
\begin{enumerate}
\item Prove that
$$
f_1 =  \underset{f\in \W\,;\,\|f\|_\W=1}{\mathrm{argmax}} \sum_{i=1}^n\langle \phi(X_i),f\rangle_\W^2
$$
may be written
$$
f_1 = \sum_{i=1}^n \alpha_1(i) \phi(X_i)\;,\quad\mbox{where}\quad \alpha_1 =  \underset{\alpha\in \rset^n\,;\, \alpha^T \bfK\alpha=1}{\mathrm{argmax}}\alpha^\top\bfK^2\alpha\;.
$$
\item Prove that $\alpha_1 = \lambda_1^{-1/2}b_1$ where $b_1$ is the unit eigenvector associated with the largest eigenvalue $\lambda_1$ of $\bfK$.
\item Write $H_d = \mathrm{span}\{f_1,\ldots,f_d\}$. Prove that, for all $1\leqslant i\leqslant n$,
$$
\pi_{H_d}(\phi(X_i)) = \sum_{j=1}^d \lambda_{j}\alpha_j(i)f_j\;.
$$
\end{enumerate}

%Let $X$ be a standard Gaussian random variable in $\rset^d$ and assume that conditionally on $X$, $Z$ has a Gaussian distribution with mean $WX + \mu$ and variance $\sigma^2 I_d$.
%\begin{enumerate}
%\item Prove that $Z$ has a Gaussian distribution with mean $\mu$ and variance $C = WW^\top + \sigma^2 I_d$.
%
%\vspace{.2cm}
%
%{\em
%It is enough to write
%$$
%Z = WX + \mu + \sigma \varepsilon\,,
%$$
%where $X$ and $\varepsilon$ are independent Gaussian random variables. Therefore, $Z \sim \mathcal{N}(\mu,C)$.
%}
%
%\item Prove that conditionally on $Z$, $X$ has a Gaussian distribution with mean $m= C^{-1}W^\top (Z-\mu)$ and variance $\Sigma = \sigma^2 C^{-1}$.
%
%\vspace{.2cm}
%
%{\em
%The joint distribution of $Z$ and $X$ can be written, for all $(x,z)\in\rset^d\times\rset^d$,
%\begin{align*}
%p(z,x) \propto \exp(-\frac{1}{2}x^\top x)\exp(-\frac{1}{2\sigma^2}(z- Wx - \mu)^\top (z- Wx - \mu))\,.
%\end{align*}
%Therefore,
%\begin{align*}
%p(x|z) &\propto  p(z,x)\\
%&\propto \exp(-\frac{1}{2}x^\top x)\exp(-\frac{1}{2\sigma^2}(z- Wx - \mu)^\top (z- Wx - \mu))\,,\\
%&\propto \exp(-\frac{1}{2\sigma^2}(z- Wx - \mu)^\top (z- Wx - \mu))\,,\\
%&\propto \exp(-\frac{1}{2\sigma^2}(x - m)^\top \Sigma^{-1} (x-m))\,,
%\end{align*}
%where $\Sigma = \sigma^2 C^{-1}$ and $m = C^{-1}W^\top (Z-\mu)$ which concludes the proof.
%}
%\item Assume that $\{Z_i\}_{1\leq i\leq n}$ are $n$ i.i.d. observations with the same distribution has $Z$. Write the loglikelihood of $(Z_1,\ldots,Z_n)$.
%
%\vspace{.2cm}
%
%{\em
%By definition,
%\begin{align*}
%\log p_C(Z_1,\ldots,Z_n) &= \sum_{i=1}^n \log p_C(Z_i)  \,,\\
%&=\sum_{i=1}^n \left(-\frac{1}{2}\log \mathrm{det}(2\pi C) - \frac{1}{2}(Z_i-m)^\top \Sigma^{-1}(Z_i-m)\right)\,,\\
%&= -\frac{n}{2}\log \mathrm{det}(2\pi C) -\frac{1}{2}\sum_{i=1}^n (Z_i-m)^\top \Sigma^{-1}(Z_i-m)\,,\\
%&= -\frac{n}{2}\log \mathrm{det}(2\pi C) -\frac{n}{2}\mathrm{Trace}(C^{-1}S_n)\,,
%\end{align*} 
%where
%$$
%S_n = \frac{1}{n}\sum_{i=1}^n(Z_i-\mu)^\top(Z_i-\mu)\,.
%$$
%}
%\item Assuming that $C = WW^\top + \sigma^2 I_d$, show that $\widehat C= \widehat{W}\widehat{W}^\top + \sigma^2 I_d$ with $\widehat{W} = U_q(\Lambda_q-\sigma^2I_q)^{1/2}R$ is a stationary point of the loglikelihood (and therefore maximizes the loglikelihood).
%
%\vspace{.2cm}
%
%{\em
%
%}
%\item 
%
%\vspace{.2cm}
%
%{\em
%
%}
%\item
%
%\vspace{.2cm}
%
%{\em
%
%}
%\end{enumerate}
\end{document}





	
